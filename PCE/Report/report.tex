\documentclass{article}
\usepackage{epsfig}
\usepackage[table]{xcolor}
\usepackage[top=0.50in, bottom=0.50in, left=0.65in, right=0.75in]{geometry}
%\usepackage[hidelinks]{hyperref}
\usepackage{hyperref}
\usepackage[utf8]{inputenc}
\usepackage[english]{babel}
\usepackage{amsmath,amssymb,amsthm}
\usepackage{booktabs} % To thicken table lines
\title{Generating Propagation Complete Encodings in Haskell \\ }
\author{\vspace{2mm} \large Arunothia Marappan \\ Under Guidance of Graeme Gange }
\date{}
\usepackage{xcolor}
\hypersetup{
    colorlinks,
    linkcolor={red!50!black},
    citecolor={blue!50!black},
    urlcolor={blue!80!black}
}
\renewcommand{\P}{\ensuremath{\textup{\textbf{P}}}}
\newcommand{\E}{\ensuremath{\textup{\textbf{E}}}}

\theoremstyle{plain}
\newtheorem{defn}{Definition}
\newtheorem{lem}{Lemma}
\newtheorem{notn}{Notation}
\newtheorem{remark}{Remark}
\newtheorem{clm}{Claim}

\begin{document}
\maketitle
\begin{center}
\emph{Submitted to Prof. Harald Sondergaard as a part of Research Internship (May'16 - July'16)}
\end{center}
\vspace*{2cm}~
\begin{abstract}
 In this project, we implemented the algorithm given in ~\cite{PCE} in Haskell.   
\end{abstract}

\begin{center}
\vspace*{7cm}~ \includegraphics[scale=1]{UOM.png} \\
\large{Computing and Information Systems \\ University of Melbourne}
\end{center}
\newpage
\nocite{*}
\bibliographystyle{unsrt}
\bibliography{ref}

\end{document}