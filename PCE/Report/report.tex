\documentclass{article}
\usepackage{epsfig}
\usepackage[table]{xcolor}
\usepackage[top=0.50in, bottom=0.50in, left=0.65in, right=0.75in]{geometry}
%\usepackage[hidelinks]{hyperref}
\usepackage{hyperref}
\usepackage[utf8]{inputenc}
\usepackage[english]{babel}
\usepackage{amsmath,amssymb,amsthm}
\usepackage{booktabs} % To thicken table lines
\title{Generating Propagation Complete Encodings in Haskell \\ }
\author{\vspace{2mm} \large Arunothia Marappan \\ Under Guidance of Graeme Gange }
\date{}
\usepackage{xcolor}
\hypersetup{
    colorlinks,
    linkcolor={red!50!black},
    citecolor={blue!50!black},
    urlcolor={blue!80!black}
}
\renewcommand{\P}{\ensuremath{\textup{\textbf{P}}}}
\newcommand{\E}{\ensuremath{\textup{\textbf{E}}}}

\theoremstyle{plain}
\newtheorem{defn}{Definition}
\newtheorem{lem}{Lemma}
\newtheorem{notn}{Notation}
\newtheorem{remark}{Remark}
\newtheorem{clm}{Claim}

\begin{document}
\maketitle
\begin{center}
\emph{Submitted to Prof. Harald Sondergaard as a part of Research Internship (May'16 - July'16)}
\end{center}
\vspace*{2cm}~
\begin{abstract}
 In this project, we are implementing the algorithm given in ~\cite{PCE} in Haskell. We define Partial Assignments as Maybe[PAValue] where PAValue is the data type that can either be True, False or Question. The ordering amongst Partial Assignments are established as - Nothing (contradicting Partial Assignment) as the least and the more the undefined the partial assignment is, higher it is in the ordering. For the implementation of Priority Queues, we use Haskell Package Data.Heap. From the ordering of Partial Assignments mentioned, it is clear that we used maxHeap to mimic the priority queue desired. Sat Solver for the implementation has been taken from ~\cite{sat.hs}.      
\end{abstract}

\begin{center}
\vspace*{7cm}~ \includegraphics[scale=1]{UOM.png} \\
\large{Computing and Information Systems \\ University of Melbourne}
\end{center}
\newpage
\nocite{*}
\bibliographystyle{unsrt}
\bibliography{ref}

\end{document}